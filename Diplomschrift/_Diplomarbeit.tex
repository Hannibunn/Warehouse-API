\documentclass[german,oneside,color]{htldipl}
% Zulässige Class Options: 
%   Hauptsprache: german (default), english
%   Doppelseitig: oneside (default), twoside
%   Syntax-Highlighting: color (default), black

% die folgende Zeile einkommentieren für Arial-Ähnliche Schriftart
%\renewcommand{\familydefault}{\sfdefault}

\graphicspath{{images/}}    % Bilderverzeichnis

\include{Settings}

\makeglossaries
\loadglsentries{glossary}					%beinhaltet Daten für das Glossar
\addbibresource{literatur.bib}     %beinhaltet Daten für das Literaturverzeichnis

%%%----------------------------------------------------------
\begin{document}
%%%----------------------------------------------------------
%Einstellungen an die eigene Diplomarbeit anpassen
\title{Sicherheit in eingebetteten Echtzeitsystemen}
\abteilung{Elektronik und\\Technische Informatik}
%\schwerpunkt{} wenn kein Ausbildungsschwerpunkt vorhanden ist z.B. Informatik
\schwerpunkt{}
\studienort{Mödling}
\schule{HTBLuVA Mödling}
\schullogo{htlmd0.jpg}
\abgabejahr{2024/25}
\betreuerA{Dr.\ Walter Turbo}
%\betreuerB{} leer lassen wenn nicht vorhanden
\betreuerB{}
\betreuerC{}
\betreuerD{}
\schuelerA{Maximilian MAIER}
\evidenzA{5AHEL-17}
\subthemaA{Angriffsvektoren und Bedrohungsanalyse in eingebetteten Systemen}
\schuelerB{Elisabeth MUSTER}
\evidenzB{5AHEL-19}
\subthemaB{Sichere Speicherverwaltung und Schutzmechanismen für Echtzeitsysteme}
\schuelerC{Peter ZAPFEL}
\evidenzC{5AHEL-24}
\subthemaC{Kryptographische Verfahren zur Sicherung von Kommunikationsschnittstellen}
\schuelerD{Otto BAUER}
\evidenzD{5BHEL-02}
\subthemaD{Absicherung von Firmware-Updates für Embedded Devices}
%\schuelerE{} leer lassen wenn nicht vorhanden
\schuelerE{}
\evidenzE{}
\subthemaE{}

%%%----------------------------------------------------------
\frontmatter
\maketitle
\tableofcontents
%%%----------------------------------------------------------

\include{vorwort}				%ggfs. weglassen
\include{dokumentation}
\include{kurzfassung}		
\include{abstract}			

%%%----------------------------------------------------------
\mainmatter           %Hauptteil (ab hier arab. Seitenzahlen)
%%%----------------------------------------------------------

\include{einleitung}
\include{diplomschrift}
\include{latex}
\include{faq}
\include{abbildungen}
\include{mathematik}
\include{literatur}
\include{drucken}
\include{word}
\include{schluss}

%%%----------------------------------------------------------
%%%Anhang
\appendix
%\include{anhang_a}	% Technische Ergänzungen
\include{anhang_b}	% Inhalt der CD-ROM/DVD
\include{anhang_c}	% Chronologische Liste der Änderungen
\include{anhang_d}	% Quelltext dieses Dokuments

%%%----------------------------------------------------------
%Ausgabe der automatischen Zusatzdaten: Glossar, Index, Literaturverzeichnis
\clearpage
\printglossaries

\clearpage
\chapter*{Index}
\addcontentsline{toc}{chapter}{Index}
\printindex[allgemein]

\printindex

\printindex[name]

\printindex[title]


%Literaturverzeichnis
\clearpage
\addcontentsline{toc}{chapter}{\bibname}

\printbibliography


%%%----------------------------------------------------------

%%%Messbox zur Druckkontrolle
%\include{messbox}

\end{document}
